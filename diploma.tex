\documentclass[a4paper,14pt]{article} % определение типа документа ,
\usepackage[14pt]{extsizes} % для того чтобы задать нестандартный 14-ый размер шрифта
\usepackage[left=3cm,right=1.5cm,top=2cm,bottom=2cm]{geometry} % поля
\usepackage[T2A]{fontenc}
\usepackage[russian]{babel}
\usepackage{indentfirst}
\usepackage{ragged2e} %выравнивание по ширине
\usepackage{setspace}
%\полуторный интервал
\begin{document}
\onehalfspacing
\setlength{\parindent}{1.25cm} %красная строка
\begin{titlepage}
\newpage
\begin{center}
\small{МИНИСТЕРСТВО НАУКИ И ВЫСШЕГО ОБРАЗОВАНИЯ РОССИЙСКОЙ ФЕДЕРАЦИИ \\
     ФЕДЕРАЛЬНОЕ ГОСУДАРСТВЕННОЕ БЮДЖЕТНОЕ \\ОБРАЗОВАТЕЛЬНОЕ УЧРЕЖДЕНИЕ 
     ВЫСШЕГО ОБРАЗОВАНИЯ\\ «СЕВЕРО-ОСЕТИНСКИЙ ГОСУДАРСТВЕННЫЙ УНИВЕРСИТЕТ \\
     ИМЕНИ КОСТА ЛЕВАНОВИЧА ХЕТАГУРОВА» \\}
\end{center}
\vspace {2em}
\begin{center}
\small{Факультет математики и компьютерных наук \\
     Кафедра прикладной математики и информатики \\
     }
\end{center}
\vspace{2em}
\begin{center}
\textbf{ВЫПУСКНАЯ КВАЛИФИКАЦИОННАЯ} \\
\textbf{РАБОТА} \\
\vspace{2em}
\textbf{Тема:} Метод опорных векторов
 \\
\end{center}
\vspace{3em}
\begin{flushright}
\vbox{%
\hfill%
\vbox{%
\hbox{\textbf{Исполнитель:}}%
\hbox{Студентка 4 курса ОФО}%
\hbox{направления подготовки «Прикладная}%
\hbox{математика и информатика»}
\hbox{Елоева В.А}
\vspace{1em}
\hbox{\textbf{Научный руководитель:}}
\hbox{к.ф.-м.н. Басаева Е.К.}
}%
} 

\end{flushright}
\vspace{1em}
\begin{flushleft}
    \textbf{«Допущена к защите»} \\
    \vspace {1em}
Заведующий кафедрой \line(1,0){106}  к.ф.-м.н. Басаева Е.К.

\end{flushleft}
\vspace{2em}
\begin{center}
Владикавказ 2023 \\
\end{center}
\end{titlepage}
\setcounter{page}{2} % со второй страницы
\newpage 
\begin{center}
    \tableofcontents
\end{center}
\newpage 
\newpage
\setcounter{subsection}{0}
\setcounter{equation}{0}
\setcounter{section}{0}

\begin{center}
\section*{Введение}
\addcontentsline{toc}{section}{Введение}
\end{center}
\parМетод опорных векторов (Support Vector Machine, SVM) – один из
популярных методов машинного обучения, который применяется для решения
задач классификации и регрессии.

Метод опорных векторов (SVM) относится к разделу машинного обучения,
называемому обучение с учителем или обучение по прецедентам. Существует четыре основных варианта метода опорных векторов:\\
\begin{enumerate}
    \item Классификатор с максимальным (жестким) зазором,
    \item Версия SVM с мягким зазором,
    \item Ядерная версия, использующая «трюк ядра»,
    \item Ядерная версия с мягким зазором, совмещающая варианты 1, 2 и 3.
\end{enumerate}
\newpage
\setcounter{subsection}{0}
\setcounter{equation}{0}
\setcounter{section}{0}

\begin{center}
\section*{Глава 1. Задача классификации}
\addcontentsline{toc}{section}{Глава 1. Задача классификации}
\end{center}
\parКлассификацией называется раздел машинного обучения, посвященный решению
следующей задачи:\\
\begin{itemize}
    \item Имеется множество «объектов» (n-мерных векторов), разделённых,
некоторым образом, на «классы».
    \item Задано конечное множество объектов, для которых известно, к каким
классам они относятся («обучающая выборка»).
    \item Классовая принадлежность остальных объектов не известна.
    \item Требуется построить алгоритм, способный классифицировать
произвольный объект из исходного множества.
\end{itemize}
\newpage
\setcounter{subsection}{0}
\setcounter{equation}{0}
\setcounter{section}{0}

\begin{center}
\section*{Глава 2. Линейно разделимый случай}
\addcontentsline{toc}{section}{Глава 2. Линейно разделимый случай}
\end{center}
\par Два класса называют линейно отделимыми, если существует хотя бы одна
гиперплоскость, разделяющая их.
\newpage
\setcounter{subsection}{0}
\setcounter{equation}{0}
\setcounter{section}{0}

\begin{center}
\section*{Глава 3. Ядра и спрямляющие пространства}
\addcontentsline{toc}{section}{Глава 3. Ядра и спрямляющие пространства}
\end{center}
Конструктивные способы построения ядер.
\begin{enumerate}
    \item Произвольное скалярное произведение K(x, z) = 〈x, z〉 является ядром.
    \item Константа K(x, z) = 1 является ядром.
\end{enumerate} 
\newpage
\setcounter{subsection}{0}
\setcounter{equation}{0}
\setcounter{section}{0}

\begin{center}
\section*{Заключение}
\addcontentsline{toc}{section}{Заключение}
\end{center}
\parОрганизационная структура предприятия может носит линейный характер. Все без исключения работники находятся в подчинении директора предприятия.
\newpage
\setcounter{subsection}{0}
\setcounter{equation}{0}
\setcounter{section}{0}

\begin{center}
\section*{Список литературы}
\addcontentsline{toc}{section}{Список литературы}
\end{center}
\begin{enumerate}
    \item Федеральный закон Российской Федерации от 14.06.1995 г. № 88-ФЗ «О государственной поддержке малого предпринимательства в Российской Федерации» (с последующими изменениями и дополнениями). 
    \item Стратегический менеджмент: основы стратегического управления / М.А. Чернышев и др. — Ростов-н/Д.: Феникс, 2019.
\end{enumerate}
\end{document}