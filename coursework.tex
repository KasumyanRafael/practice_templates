\documentclass[14pt, a4paper]{extarticle}
%%%%%%%%%%%%%%%%% Оформление ГОСТА%%%%%%%%%%%%%%%%%

% Все параметры указаны в ГОСТЕ на 2021, а именно:

% Шрифт для курсовой Times New Roman, размер – 14 пт.



% шрифт для URL-ссылок
%\urlstyle{same} 

% Междустрочный интервал должен быть равен 1.5 сантиметра.
\linespread{1.5} % междустрочный интервал


% Каждая новая строка должна начинаться с отступа равного 1.25 сантиметра.
\setlength{\parindent}{1.25cm} % отступ для абзаца


% Текст, который является основным содержанием, должен быть выровнен по ширине по умолчанию включен из-за типа документа в main.tex


%%%%%%%%%%%%%%%%%% Дополнения %%%%%%%%%%%%%%%%%%%%%%%%%%%%%%%%%

% Путь до папки с изображениями


% Внесение titlepage в учёт счётчика страниц
\makeatletter
\renewenvironment{titlepage} {
	\thispagestyle{empty}
}


% Цвет гиперссылок и цитирования
\usepackage{hyperref} 
 \hypersetup{ 
     colorlinks=true, 
     linkcolor=black, 
     filecolor=blue, 
     citecolor = black,       
     urlcolor=blue, 
     }
    

% Нумерация рисунков
\counterwithin{figure}{section}

% Нумерация таблиц
\counterwithin{table}{section}

\counterwithin{table}{section}

% шрифт для листингов с лигатурами


% Перенос текста при переполнении
\emergencystretch=25pt


% настройка подсветки кода и окружения для листингов
%\usemintedstyle{colorful} % делает подсветку для кода
\newenvironment{code}{\captionsetup{type=listing}}{}


% Посмотреть ещё стили можно тут https://www.overleaf.com/learn/latex/Code_Highlighting_with_minted
\usepackage[14pt]{extsizes} % для того чтобы задать нестандартный 14-ый размер шрифта
\usepackage{polyglossia} % языковой пакет
\usepackage{amsmath} % поддержка математических символов
\usepackage{url} % поддержка url-ссылок
\usepackage{indentfirst}
\usepackage[left=3cm,right=1.5cm,top=2cm,bottom=2cm]{geometry} % поля
\usepackage{setspace}
%\полуторный интервал
\usepackage{ragged2e} %выравнивание по ширине
\setlength{\parindent}{1.25cm} %красная строка
\begin{document}
\onehalfspacing   
\setmonofont{Times New Roman}
\setmainfont{Times New Roman} 
\setromanfont{Times New Roman} 
\newfontfamily\cyrillicfont{Times New Roman}
\newfontfamily\cyrillicfont{Times New Roman}
\begin{titlepage}
\newpage
\begin{center}
\small{МИНИСТЕРСТВО НАУКИ И ВЫСШЕГО ОБРАЗОВАНИЯ РОССИЙСКОЙ ФЕДЕРАЦИИ \\
     ФЕДЕРАЛЬНОЕ ГОСУДАРСТВЕННОЕ БЮДЖЕТНОЕ ОБРАЗОВАТЕЛЬНОЕ \\
     УЧРЕЖДЕНИЕ ВЫСШЕГО ОБРАЗОВАНИЯ \\
     «СЕВЕРО-ОСЕТИНСКИЙ ГОСУДАРСТВЕННЫЙ УНИВЕРСИТЕТ \\
     ИМЕНИ КОСТА ЛЕВАНОВИЧА ХЕТАГУРОВА» \\}
\end{center}
\vspace {2em}
\begin{center}
\small{Факультет математики и компьютерных наук \\
     Кафедра прикладной математики и информатики \\
     }
\end{center}
\vspace{2em}
\begin{center}
\textbf{КУРСОВАЯ РАБОТА} \\ 
\vspace{2em}
\textbf{По курсу: Основы и методология программирования} \\
\end{center}
\vspace{1.1em}
\begin{center}
Тема: «Осетинское лото (языковая игра)»
\end{center}
\vspace{3em}
\begin{flushright}
      \vbox{%
\hfill%
\vbox{%
\hbox{\textbf{Исполнитель:} cтудент 2-ПМ ОФО}%
\hbox{Касумян Рафаэль Ефремович}%
\hbox{\textbf{Руководитель:} Макаренко М.Д.}
}%
} 
\end{flushright}
\vspace{8em}
\begin{center}
Владикавказ 2023 \\
\end{center}
\end{titlepage}
\setcounter{page}{1} % со второй страницы
\newpage 
\begin{center}
    \tableofcontents
\end{center}
\newpage
\setcounter{subsection}{0}
\setcounter{equation}{0}
\setcounter{section}{0}
\begin{center}
\section*{Введение}
\addcontentsline{toc}{section}{Введение}
\end{center}
\par Мы живём в Республике Северная Осетия-Алания – одном из \\ многонациональных регионов нашей необъятной страны. Осетинский народ имеет очень богатую историю, свои традиции, обычаи и культуру. Вместе с тем, время не стоит на месте, поэтому многое постепенно забывается, стирается из памяти под воздействием современных реалий. Это отнюдь не благоприятное явление коснулось и осетинского языка - одного из богатейших языков народов Кавказа...
\newpage
\setcounter{subsection}{0}
\setcounter{equation}{0}
\setcounter{section}{0}

\begin{center}
\section*{Глава 1. Теоретическая часть}
\addcontentsline{toc}{section}{Глава 1. Теоретическая часть}
\end{center}
\par Игра, рассчитанная на двух игроков, состоит из трёх раундов, по результатам которых определяется победитель. Первые два из них \\представлены ниже: каждому из игроков будет отведено по клетчатому полю задаваемого в настройках размера, ячейки которых после загрузки будут заполнены изображениями (1. 2 раунды) или словами из осетинского языка (3 раунд). По прошествии времени, задаваемым настраиваемым таймером, в окне будут появляться картинки...
\newpage
\setcounter{subsection}{0}
\setcounter{equation}{0}
\setcounter{section}{0}

\begin{center}
\section*{Глава 2. Практическая часть}
\addcontentsline{toc}{section}{Глава 2. Практическая часть}
\end{center}
\par Работу программы можно разделить на следующие этапы: 
\begin{enumerate}
    \item Запоминание имён игроков 
    \item Подсчёт очков и выявление победителя (в каждом раунде) 
    \item Демонстрация таблицы с результатами игроков по раундам, выявление победителя всей игры. 
\end{enumerate}
\par При создании игры были подгружены изображения, которые должны появляться в таблицах игроков. Для правильной работы с ними был создан класс Word, в который подаётся строковый массив из файла со словами и переводами...
\newpage
\setcounter{subsection}{0}
\setcounter{equation}{0}
\setcounter{section}{0}

\begin{center}
\section*{Заключение}
\addcontentsline{toc}{section}{Заключение}
\end{center}
\par Благодаря данному проекту я освоил компонент DataGridView(таблицу), научился в программе регулировать её размеры, также освоил метод добавления картинок в данную структуру. Помимо этого, я научился работать с классами и самостоятельно продумывать их функционал, добавлять поля, методы и свойства...
\newpage
\setcounter{subsection}{0}
\setcounter{equation}{0}
\setcounter{section}{0}

\begin{center}
\section*{Список литературы}
\addcontentsline{toc}{section}{Список литературы}
\end{center}
\begin{enumerate}
    \item Сайт Microsoft, обучающий работе с Visual Studio \par \url{https://docs.microsoft.com/ru-ru/dotnet/desktop/winforms/controls/}
    \item Онлайн форум stack overflow \url{https://ru.stackoverflow.com/}
    \item Онлайн форум CyberForum.ru \url{https://www.cyberforum.ru/}
    \item Веб сайт “Хабр” \url{https://habr.com/}
\end{enumerate}
\end{document}