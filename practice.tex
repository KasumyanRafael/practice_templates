\documentclass[a4paper,12pt]{article} % определение типа документа ,
\usepackage[14pt]{extsizes} % для того чтобы задать нестандартный 14-ый размер шрифта
\usepackage[left=3cm,right=1.5cm,top=2cm,bottom=2cm]{geometry} % поля
\usepackage[T2A]{fontenc}
\usepackage{ragged2e} %выравнивание по ширине
\usepackage{indentfirst}
\usepackage[russian]{babel}
\usepackage{setspace}
%\полуторный интервал
\begin{document}
\setlength{\parindent}{1.25cm} %красная строка
\onehalfspacing
\begin{titlepage}
\newpage
\begin{center}
\small{ФГБОУ ВО «СЕВЕРО-ОСЕТИНСКИЙ ГОСУДАРСТВЕННЫЙ УНИВЕРСИТЕТ  \\
     ИМЕНИ К.Л. ХЕТАГУРОВА» \\}
\end{center}
\vspace {1em}
\begin{center}
\small{Факультет математики и компьютерных наук \\
     Кафедра прикладной математики и информатики \\
     }
\end{center}
\vspace{1em}
\begin{center}
\textbf{ОТЧЁТ ПО ПРАКТИКЕ} \\ 
\end{center}
\vspace{1.1em}
\begin{flushleft}
Наименование практики: \underline{Учебная (технологическая (проектно-технологическая))
практика} \\
\vspace{1em}
направление подготовки\underline{ 01.03.02 Прикладная математика и информатика,} \\
профиль\underline{ «Программирование, анализ данных и математическое моделирование»} \\
\vspace{2em}
Выполнил(и) студент(ы) \begin{tabular}{lp{2em}l} 
    \hspace{3cm}   && \hspace{2cm}  \\\cline{1-1}\cline{3-3}          
    (ФИО)     && (подпись)
  \end{tabular} \\
  \begin{tabular}{lp{2em}l} 
    \hspace{3cm}   && \hspace{2cm}  \\\cline{1-1}\cline{3-3}          
    (ФИО)     && (подпись)
  \end{tabular}\\
  \vspace{2em}
  Дата сдачи отчёта: \underline{«14» июля 2023 г.}\\
  \vspace{2em}
  Отчёт принят: \begin{tabular}{lp{2em}l} 
    \hspace{3cm}  && \hspace{2cm}  \\\cline{1-1}\cline{3-3}          
    (подпись)     && Ф.И.О. преподавателя-экзаменатора
  \end{tabular} 
\end{flushleft}
\end{titlepage}
\setcounter{page}{2} % со второй страницы
\newpage 
\begin{center}
    \tableofcontents
\end{center}
\newpage 
\newpage
\setcounter{subsection}{0}
\setcounter{equation}{0}
\setcounter{section}{0}
\begin{center}
\section*{Введение}
\addcontentsline{toc}{section}{Введение}
\end{center}
\par \textbf{Целями} Учебной практики (технологическая (проектно-технологическая)) является профессионально-компетентностная подготовка обучающихся к самостоятельной профессиональной деятельности посредством формирования навыков и иных компетенций, опыта самостоятельной профессиональной деятельности. 

\textbf{Задачи Практики:} 
\begin{itemize}
    \item закрепление, углубление и систематизация знаний, полученных при изучении дисциплин профессионального цикла;
    \item развитие имеющихся и приобретение новых профессиональных умений и навыков; 
    \item развитие сформированных и формирование новых компетенций по избранной профессиональной деятельности;
    \item укрепление связи обучения с практической деятельностью.
\end{itemize}
\par \textbf{Место прохождения практики:} кафедра прикладной математики и информатики.
\par \textbf{Форма проведения Практики:} непрерывно.
\par \textbf{Способ проведения Практики:} стационарная.
\par \textbf{Период прохождения практики:} с 03.07.2023 по 15.07.2023 включительно (2 недели).
\par Руководство практикой осуществлялось доцентом кафедры прикладной математики и информатики Гутновой А.К., старшим преподавателем кафедры прикладной математики и информатики Дзанаговой И.Т. 
\parПрохождение практики осуществлялось в соответствии с рабочим графиком (см. приложение 1) и индивидуальным заданием (см. приложение 2).
\par \textbf{Краткое содержание и цель индивидуального задания:}\\
…………………………………………………………………………… ……………………………………………….……………….………………... \\
\par \textbf{Перечень выполненных работ и заданий.} В соответствии с индивидуальным заданием за период прохождения практики выполнена следующая работа: \\
………………………………………………………………………….. \\
………………………………………………………………………….. \\
…………………………………………………………………………. \\
\parВсе виды деятельности в период прохождения практики отражены в Дневнике практики (см. приложение 3). \\ 
\newpage
\setcounter{subsection}{0}
\setcounter{equation}{0}
\setcounter{section}{0}

\begin{center}
\section*{Название в соответствии с заданием на практику}
\addcontentsline{toc}{section}{Название в соответствии с заданием на практику}
\end{center}
\par Постановка задачи 
\par Методы и алгоритмов ее решения,
\par Этапы выполненных работ 
\par Результаты решения задачи
\newpage
\setcounter{subsection}{0}
\setcounter{equation}{0}
\setcounter{section}{0}

\begin{center}
\section*{Заключение}
\addcontentsline{toc}{section}{Заключение}
\end{center}
\par Описание (характеристика) навыков и умений, приобретенных на практике. 
\par Индивидуальные выводы о практической значимости навыков, приобретенных на практике. 
\newpage
\setcounter{subsection}{0}
\setcounter{equation}{0}
\setcounter{section}{0}

\begin{center}
\section*{Список литературы}
\addcontentsline{toc}{section}{Список литературы}
\end{center}
1 источник \\
2 источник \\
\end{document}